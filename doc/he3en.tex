\documentclass[a4paper]{article}

\usepackage{amssymb}
\usepackage{euscript}
\usepackage{graphicx}
\graphicspath{{pics/}}
\usepackage{color}
\usepackage{epsfig}
\usepackage{fullpage}

\newcommand{\image}[3]{
\begin{figure}[#1]
\begin{center}
\includegraphics{full_#2.eps}
\caption{\small#3}
\label{image:#2}
\end{center}
\end{figure}
}

\def\ts{{\theta^s}}
\def\ta{{\theta^A}}
\def\tb{{\theta^B}}
\def\ct{\cos\vartheta}
\def\st{\sin\vartheta}
\def\cts{\cos^2\vartheta}
\def\sts{\sin^2\vartheta}

\begin{document}

This is a collection of matlab/octave scripts for testing various $^3$He-B
energies. Energies are calculated using the order parameter rotation matrix $R$ or its
representation via rotation axis and angle, $\vartheta$ and~$\bf n$. Unit vector $\bf n$
can be represented using its azimuthal and polar angles~$\alpha$ and~$\beta$.

A small rotation $\ts$ can be applied to the matrix. This means rotation of the spin space
at~$|\ts|$ angle around $\hat\ts$ axis.

{\bf 1. Small rotation.}
{\tt test\_rot1.m} script checks equivalence of the functions:
\begin{eqnarray*}
\mbox{\tt rot\_th0(R,ths)} &=& R_{ab}({\bf \ts}) R_{bj}\\
\mbox{\tt rot\_th1(R,ths)} &=& 
\left(\delta_{ab} - e_{abc}\ts_c + \frac12 \ts_a \ts_b
 - \frac12 \delta_{ab}\ts_c \ts_c
\right) R_{bj} + O(|\ts|^3)
\end{eqnarray*}

{\bf 2. Double rotation.}
{\tt test\_rot2.m} script checks the expersion:
$$
R_{ab}(\tb)R_{bc}(\ta) R^0_{cj} \quad=\quad
R_{ab}\left(\ta + \tb + \frac12 \ta\times\tb + O(\ta^3,\tb^3)\right) R^0_{bj}
$$

{\bf 3. Dipolar energy.}
{\tt test\_d.m} script checks equivalence of the functions:
\begin{eqnarray*}
\mbox{\tt en\_d0(R)} &=& F_D(R) = R_{jj}R_{kk} + R_{jk}R_{kj}\\[2mm]
\mbox{\tt en\_d1(R)} &=& \mbox{Tr} (R^2) + (\mbox{Tr} R)^2\\
\mbox{\tt en\_d2(R)} &=& \frac12 (1 + 4\ct)^2 - \frac12
\end{eqnarray*}

{\bf 4. Dipolar energy after a small rotation.}
{\tt test\_dr.m} script checks equivalence of the functions:

\begin{eqnarray*}
\mbox{\tt en\_dr0(R,ths)} &=&
F_D\left(R_{ab}({\bf \ts}) R_{bj}\right) + O(|\ts|^3)\\[2mm]
\mbox{\tt en\_dr1(R,ths)} &=&
[ R_{jj} R_{kk} + R_{jk}R_{kj} ]\ (1-|\ts|^2)\\
&+& [R_{aj} R_{kk} + R_{ak} R_{kj}]\ (\ts_j \ts_a - 2e_{ja a'}\ \ts_{a'}) \\
&+& [R_{aj} R_{bk} + R_{ak} R_{bj}]\ e_{jaa'}\ \ts_{a'}\ e_{kbb'}\ \ts_{b'}
\\[2mm]
\mbox{\tt en\_dr2(R,ths)} &=&
 - \frac12 + \frac12 (4\ct+1)^2\\
 &-& 4(4\ct+1)\st\ ({\bf n}\cdot{\bf\ts})\\
 &-& (4\ct^2+5\ct+1)\ |{\bf\ts}|^2\\
 &+& (9 + 3\ct - 12\ct^2)\ ({\bf n}\cdot{\bf\ts})^2
\end{eqnarray*}

{\bf 5. Gradient energy.} In the gradient energy expression:
$$
F_\nabla \label{eq:He3_en_g}
= \frac12 \Delta^2 \left[
  K_1 (\nabla_j R_{ak})(\nabla_j R_{ak})
+ K_2 (\nabla_j R_{ak})(\nabla_k R_{aj})
+ K_3 (\nabla_j R_{aj})(\nabla_k R_{ak}) \right]
$$
three terms are calculated and compared separately.

{\tt test\_g.m} script compares gradient energy, calculated using following
expressions for the $\nabla R$:

\begin{eqnarray*}
\mbox{\tt en\_g0(a,b,th, ga,gb,gt, dx):} &&\\
\nabla_k R_{aj} &=&
\frac{1}{dx}\left[
R_{aj}(\alpha+\nabla_k\alpha\ dx,\ \beta+\nabla_k\beta\ dx,\ \vartheta+\nabla_k\vartheta\ dx) -
R_{aj}(\alpha, \beta, \vartheta)\right]\\
&+& O(dx^2)
\\
\mbox{\tt en\_g1(a,b,th, ga,gb,gt):} &&\\
\nabla_k R_{aj} &=&
 [(1-\ct)(\delta_{al} n_j + \delta_{jl} n_a) - \st\ e_{ajl}]\ \nabla_k n_l\\
&+& [\st\ (n_a n_j - \delta_{aj}) - \ct\ e_{ajl} n_l]\ \nabla_k \vartheta
\end{eqnarray*}

{\bf 6. Gradient energy after a small rotation.} 

{\tt test\_gr.m} script compares gradient energy, calculated using following
expressions:

\begin{eqnarray*}
\mbox{\tt en\_gr0(a,b,th, ga,gb,gt, th,gth):} &&\mbox{direct calculation with} \\
\nabla (R(\ts) R^0) &=& R(\ts) (\nabla R^0) + (\nabla R(\ts)) R^0
\\
\mbox{\tt en\_gr1(a,b,th, ga,gb,gt, th,gth):} &&\\
   F_\nabla\ =\ F_\nabla^0
&+& (2K_1+K_2+K_3)
\ (\nabla_j \ts_a) (\nabla_j \ts_a)\\
&-& [K_2 R^0_{aj} R^0_{bk} + K_3 R^0_{ak} R^0_{bj}]
\ (\nabla_j \ts_a) (\nabla_k \ts_b)\\
&+& [K_1 R^0_{ak} (\nabla_j R^0_{bk}) + K_2 R^0_{ak} (\nabla_k R^0_{bj})
   + K_3 R^0_{aj} (\nabla_k R^0_{bk})]\times\\
&&\times(2e_{abc} + \delta_{ac}\ts_b - \delta_{bc}\ts_a)\ (\nabla_j \ts_c) + O(|\ts|^3)
\end{eqnarray*}



%  % - (K2 R_aj R_bk + K3 R_ak R_bj) (\nabla_j th_a)(\nabla_k th_b)
%  for k=1:3; for j=1:3;
%    for a=1:3; for b=1:3;
%      e2 = e2 - r0(a,j)*r0(b,k)*gth(a,j)*gth(b,k);
%      e3 = e3 - r0(a,k)*r0(b,j)*gth(a,j)*gth(b,k);
%    end end
%  end; end
%
%  %
%  for k=1:3; for j=1:3;
%    for a=1:3; for b=1:3; for c=1:3;
%      xx = (2*ee(a,b,c) + dd(c,a)*th(b) - dd(c,b)*th(a))*gth(c,j);
%      e1 = e1 + xx*r0(a,k)*gr0(b,k,j);
%      e2 = e2 + xx*r0(a,k)*gr0(b,j,k);
%      e3 = e3 + xx*r0(a,j)*gr0(b,k,k);
%    end; end; end;
% end; end;

{\bf 7. Gradient torque in $\bf n$ and $\theta$ coordinates}

{\tt test\_torque.m} script builds a rotation matrix with
random first and second-order gradients on a $3\times3\times3$ grid.
Formulas for $R_{aj}$, $\nabla_kR_{aj}$, $\nabla_l\nabla_kR_{aj}$ are checked:
\begin{eqnarray}
R_{aj} &=& \ct\ \delta_{aj} + (1-\ct)\ n_a n_j - \st\ e_{ajk}n_k
\end{eqnarray}
\begin{eqnarray}
\nabla_k R_{aj} &=&
\st\ (n_a n_j - \delta_{aj})\ \nabla_k\vartheta
\\\nonumber &&
+ (1-\ct)\ (n_j \nabla_k n_a + n_a \nabla_k n_j)
\\\nonumber &&
- \ct\ e_{ajm} n_m\ \nabla_k\vartheta
\\\nonumber &&
- \st\ e_{ajm} \nabla_k n_m
\end{eqnarray}
\begin{eqnarray}
\nabla_l \nabla_k R_{aj} &=&
+ \ct\ (n_a n_j - \delta_{aj})\ \nabla_l\vartheta\ \nabla_k\vartheta
\\\nonumber &&
+ \st\ (n_j \nabla_l n_a + n_a \nabla_l n_j)\ \nabla_k\vartheta
\\\nonumber &&
+ \st\ (n_a n_j - \delta_{aj})\ \nabla_l\nabla_k\vartheta
\\\nonumber &&
+ (1-\ct)\ (\nabla_l n_j \nabla_k n_a + \nabla_l n_a \nabla_k n_j)
\\\nonumber &&
+ (1-\ct)\ (n_j \nabla_l \nabla_k n_a + n_a \nabla_l \nabla_k n_j)
\\\nonumber &&
+ \st\ (n_j \nabla_k n_a + n_a \nabla_k n_j)\nabla_l\vartheta
\\\nonumber &&
+ \st\ e_{ajm}n_m\ \nabla_l\vartheta\ \nabla_k\vartheta
\\\nonumber &&
- \ct\ e_{ajm}\nabla_l n_m\ \nabla_k\vartheta
\\\nonumber &&
- \ct\ e_{ajm}n_m\ \nabla_l \nabla_k\vartheta
\\\nonumber &&
- \ct\ e_{ajm} \nabla_k n_m \nabla_l\vartheta
\\\nonumber &&
- \st\ e_{ajm} \nabla_l \nabla_k n_m
\end{eqnarray}

Then two terms of the gradient torque is checked:
\begin{eqnarray}
T_a^\nabla
 &=& - e_{abc} R_{cj}
  \Delta^2 \big[
     K_1 (\nabla_k \nabla_k R_{bj})
    + (K_2+K_3) (\nabla_k \nabla_j R_{bk}) \big]
\end{eqnarray}

Result is too heavy, more then~40 terms.

\end{document}


